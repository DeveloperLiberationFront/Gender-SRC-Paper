%-----------------------------------------------------------------------------
%
%               Template for sigplanconf LaTeX Class
%
% Name:         sigplanconf-template.tex
%
% Purpose:      A template for sigplanconf.cls, which is a LaTeX 2e class
%               file for SIGPLAN conference proceedings.
%
% Guide:        Refer to "Author's Guide to the ACM SIGPLAN Class,"
%               sigplanconf-guide.pdf
%
% Author:       Paul C. Anagnostopoulos
%               Windfall Software
%               978 371-2316
%               paul@windfall.com
%
% Created:      15 February 2005
%
%-----------------------------------------------------------------------------


\documentclass{sigplanconf}

% The following \documentclass options may be useful:

% preprint      Remove this option only once the paper is in final form.
% 10pt          To set in 10-point type instead of 9-point.
% 11pt          To set in 11-point type instead of 9-point.
% authoryear    To obtain author/year citation style instead of numeric.

\usepackage{amsmath}
\usepackage{natbib}


\begin{document}

\special{papersize=8.5in,11in}
\setlength{\pdfpageheight}{\paperheight}
\setlength{\pdfpagewidth}{\paperwidth}

\conferenceinfo{SPLASH '15}{October 25--30, 2015, Pittsburgh, PA, United States} 
\copyrightyear{20yy} 
\copyrightdata{978-1-nnnn-nnnn-n/yy/mm} 
\doi{nnnnnnn.nnnnnnn}

% Uncomment one of the following two, if you are not going for the 
% traditional copyright transfer agreement.

%\exclusivelicense                % ACM gets exclusive license to publish, 
                                  % you retain copyright

%\permissiontopublish             % ACM gets nonexclusive license to publish
                                  % (paid open-access papers, 
                                  % short abstracts)

\titlebanner{banner above paper title}        % These are ignored unless
\preprintfooter{short description of paper}   % 'preprint' option specified.

\title{Contributions of the Under-Appreciated}
\subtitle{Gender Bias in an Open-Source Ecology}

\authorinfo{Andrew Kofink}
           {North Carolina State University}
           {ajkofink@ncsu.edu}
% \authorinfo{Name2\and Name3}
%            {Affiliation2/3}
%            {Email2/3}

\maketitle

\begin{abstract}
Sed, tempus tincidunt, pulvinar a, lectus. Vestibulum ante ipsum primis in
faucibus orci luctus et ultrices posuere cubilia Curae; Maecenas interdum purus
id risus. Ut ultricies cursus dui. In nec enim at odio aliquam iaculis. Fusce
nisl. Pellentesque sagittis. Lorem ipsum dolor sit amet, consectetuer adipiscing
elit. Aenean placerat tellus. In semper sagittis enim. Aliquam risus neque,
pretium in, fermentum vitae, vulputate et, massa. Nulla sed erat vel eros ornare
venenatis.
\end{abstract}

% \category{CR-number}{subcategory}{third-level}

% general terms are not compulsory anymore, 
% you may leave them out
% \terms
% term1, term2

\keywords
gender bias, software, open source

\section{Problem and Motivation}
Many human cultures perceive female contributions to technical
fields as inappropriate for religious or culturally constructed reasons with
negligible rationale. While personal freedom is valued in many states, the
resulting economic inefficiency is a severe handicap. Furthermore, by any modern
deontology, human females are due equal consideration as human males.

Blatant under-appreciation of an individual or group in a certain
discipline can discourage that group's contributions, further enforcing a
constructed bias for that discipline.

We believe that the contributions of female open source software developers are
likely to be accepted and merged into collaborative projects on GitHub, an
open-source software repository, at a different ratio than the contributions
of male counterparts. Humans can determine the gender of a user many different
ways, including inspection of a first and last name, with accuracy. Achieving
similar accuracy with a computer program is non-trivial. Using recent publications in automated gender
identification, in tandem with an application programmable interface (API) to
access user data, we can compute the ratio of merged pull requests to
determine the success ratio for each gender. If the difference of female to male merge success is
significant, it may provide insight in understanding how gender differences are
perceived and acted upon in subsets of
society; it will also show how to repeat this process for larger, more extensive
studies, suggested in future work.

\section{Background and Related Work}
Many collaborative sites require minimal user information; for GitHub, only a valid email
address and a user name is necessary. This creates anonymity but also creates a
challenge in the automation of gender-based studies, such
as this one. Recent work to decipher gender information from user name data has been
published by \cite{hemphill2014feminism}. This work enables an automated study
of over $1e5$ raw user accounts in this study and many more in the future.

\section{Approach and Uniqueness}
We will obtain user and gender information mined in November 2014 from Bogdan
Nicolae Vasilescu. Vasilescu used \cite{hemphill2014feminism} to determine the
gender of each user based on the user first and last name.

\section{Results and Contributions}
We analyzed N pull requests with determined gender information from M different
users. There were X pull requests created by female contributers and Y made by
male. Of the X created by female, X' were successfully merged ($\frac{X'}{X}$). Y' male-created pull requests
were merged successfully ($\frac{Y'}{Y}$).

In this study, we used GitHub as a public source of real-world behavioral data.
Many other collaborative, public environments exist on the Internet, complete
with programmatic access: Wikipedia, Stack Overflow, etc. These other sites
represent different mutable, mutually inclusive subgroups of the global
society and can be used to understand behavioral tendencies of those subgroups
towards gender perception.

% \appendix
% \section{Appendix Title}
% This is the text of the appendix, if you need one.

\acks

% Acknowledgments, if needed.

% We recommend abbrvnat bibliography style.

\bibliographystyle{abbrvnat}
\renewcommand{\bibfont}{\normalsize}

% The bibliography should be embedded for final submission.

\begin{thebibliography}
\softraggedright

\bibitem[Hemphill, Libby, et. al. (2014)]{hemphill2014feminism}

\end{thebibliography}


\end{document}

%                       Revision History
%                       -------- -------
%  Date         Person  Ver.    Change
%  ----         ------  ----    ------

%  2013.06.29   TU      0.1--4  comments on permission/copyright notices

