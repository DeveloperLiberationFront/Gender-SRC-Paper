%-----------------------------------------------------------------------------
%
%               Template for sigplanconf LaTeX Class
%
% Name:         sigplanconf-template.tex
%
% Purpose:      A template for sigplanconf.cls, which is a LaTeX 2e class
%               file for SIGPLAN conference proceedings.
%
% Guide:        Refer to "Author's Guide to the ACM SIGPLAN Class,"
%               sigplanconf-guide.pdf
%
% Author:       Paul C. Anagnostopoulos
%               Windfall Software
%               978 371-2316
%               paul@windfall.com
%
% Created:      15 February 2005
%
%-----------------------------------------------------------------------------


\documentclass{sigplanconf}

% The following \documentclass options may be useful:

% preprint      Remove this option only once the paper is in final form.
% 10pt          To set in 10-point type instead of 9-point.
% 11pt          To set in 11-point type instead of 9-point.
% authoryear    To obtain author/year citation style instead of numeric.

\usepackage{amsmath}
\usepackage{natbib}


\begin{document}

\special{papersize=8.5in,11in}
\setlength{\pdfpageheight}{\paperheight}
\setlength{\pdfpagewidth}{\paperwidth}

\conferenceinfo{SPLASH '15}{October 25--30, 2015, Pittsburgh, PA, United States} 
\copyrightyear{20yy} 
\copyrightdata{978-1-nnnn-nnnn-n/yy/mm} 
\doi{nnnnnnn.nnnnnnn}

% Uncomment one of the following two, if you are not going for the 
% traditional copyright transfer agreement.

%\exclusivelicense                % ACM gets exclusive license to publish, 
                                  % you retain copyright

%\permissiontopublish             % ACM gets nonexclusive license to publish
                                  % (paid open-access papers, 
                                  % short abstracts)

\titlebanner{banner above paper title}        % These are ignored unless
\preprintfooter{short description of paper}   % 'preprint' option specified.

\title{Contributions of the Under-Appreciated}
\subtitle{Gender Bias in an Open-Source Ecology}

\authorinfo{Andrew Kofink}
           {North Carolina State University}
           {ajkofink@ncsu.edu}
% \authorinfo{Name2\and Name3}
%            {Affiliation2/3}
%            {Email2/3}

\maketitle

\begin{abstract}
\end{abstract}

% \category{CR-number}{subcategory}{third-level}

% general terms are not compulsory anymore, 
% you may leave them out
% \terms
% term1, term2

\keywords
gender bias, software, open source

\section{Problem and Motivation}
Many human cultures perceive female contribution to technical
fields as inappropriate due to religious or culturally constructed attitudes, with
negligible rationale. This has contributed to a gross imbalance in gender diversity among
the most technical computing and engineering fields, stymieing economic,
scientific, and innovative pursuits.

The resulting inefficiencies and creative
squander can be a severe handicap to any society, notwithstanding universal human rights
violations. By any modern deontological theory, human females are due equal
consideration as human males, ethically and professionally.

Blatant under-appreciation of an individual or group in a certain
discipline can discourage that group's contributions, further enforcing a
constructed bias for that discipline; software development is one such
field.

Women account for only X\% of professional software developers, with little
evidence of improving, with female undergraduate computer science students composing Y\% of
their classes on average. This gender unbalance could be caused by many troubling subtleties
in societies, one potential being the differences, if any, in interactions
between females and males, via any medium.

These interactions include asynchronous web-based software development
collaboration. If there are significant differences in the treatment given to males
and females, overall success in the collaborative project contribution process
of GitHub, an open-source software repository host, may be one indication.

\section{Background and Related Work}
Humans can determine the gender of another individual via many different
ways, including inspection of a first and last name, with accuracy. Achieving
similar accuracy with a computer program is non-trivial; however, contacting the
author of a recent publication in automated gender identification, we were
granted access to user data mined from GitHub's GHTorrent in November 2014,
along with the gender of all GitHub users up to that date.

In tandem with GHTorrent's application programmable interface (API) to
access user pull request history, the user-gender data allows the computation of
the ratio of merge success for each gender.

Regardless of the significance in the difference of female to male success, this study will 
show how to repeat this process for other public data. In this study, we use GitHub as a public source of real-world behavioral data.
Many other collaborative, public environments exist on the Internet, complete
with programmatic access: Wikipedia, Stack Overflow, etc. These other sites
represent different mutable, mutually inclusive subgroups of the global
society and can be used to understand behavioral tendencies of those subgroups
towards gender perception.

\section{Approach and Uniqueness}
Due to the politically volatile nature of this and other related data on gender
or diversity distributions in a service's use, many organizations which value
public opinion are apprehensive about advertising gender inequalities. These new data
mining methods advance the reach of social science into non-obtrusive public
ecosystems of the internet without exposing a company's prior knowledge of a
potential issue.

\section{Results and Contributions}
We analyzed 3.8 million pull requests with determined gender information from
2.9 million different users. There were 197936 pull requests created by female contributers and 3651803 made by
male. Of the 197936 created by female, 88884 were successfully merged
($44.91\%$), and 1620168 male-created pull requests
were merged successfully ($44.37\%$).

While female pull request contributions account for only $5.14\%$ of the total pull
requests submitted in this data set, it is surprising to find that the merge
success rate is nearly equivalent to male success rate. Unequal gender
distribution found in technical fields may yet be worsened by unequal treatment;
however, the public proceedings of GitHub users, at least until November 2014,
showed no bias in female collaborative success, despite the 8.71\% female
user base represented by that data.

% \appendix
% \section{Appendix Title}
% This is the text of the appendix, if you need one.

\acks

% Acknowledgments, if needed.

% We recommend abbrvnat bibliography style.

\bibliographystyle{abbrvnat}
\renewcommand{\bibfont}{\normalsize}

% The bibliography should be embedded for final submission.

\begin{thebibliography}
\softraggedright

\bibitem{elamin2010saudiwomen}

\end{thebibliography}


\end{document}

%                       Revision History
%                       -------- -------
%  Date         Person  Ver.    Change
%  ----         ------  ----    ------

%  2013.06.29   TU      0.1--4  comments on permission/copyright notices

:!
