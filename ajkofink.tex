
\documentclass{sigplanconf}

\usepackage{amsmath}
\usepackage{natbib}
\usepackage[scientific-notation=true]{siunitx}
\usepackage{graphicx}

\begin{document}

\special{papersize=8.5in,11in}
\setlength{\pdfpageheight}{\paperheight}
\setlength{\pdfpagewidth}{\paperwidth}

\conferenceinfo{SPLASH '15}{October 25--30, 2015, Pittsburgh, PA, United States}
\copyrightyear{2015}
\copyrightdata{978-1-nnnn-nnnn-n/yy/mm}
\doi{nnnnnnn.nnnnnnn}

% Uncomment one of the following two, if you are not going for the 
% traditional copyright transfer agreement.

%\exclusivelicense                % ACM gets exclusive license to publish, 
                                  % you retain copyright

%\permissiontopublish             % ACM gets nonexclusive license to publish
                                  % (paid open-access papers, 
                                  % short abstracts)

\titlebanner{banner above paper title}        % These are ignored unless
\preprintfooter{short description of paper}   % 'preprint' option specified.

\title{Contributions of the Under-Appreciated}
\subtitle{Gender Bias in an Open-Source Ecology}

\authorinfo{Andrew Kofink}
           {North Carolina State University}
           {ajkofink@ncsu.edu}

\maketitle

\begin{abstract}
  Female software developers account for only a small portion of the total
  developer community. This inequality is caused by subtle beliefs and 
  sometimes interactions between different genders and society,
  referred to as implicit biases and explicit behavior, respectively.
  In this study, I mined user contribution acceptance from a popular software collaboration
  service. The contributions of female developers
  were accepted into open-source projects with roughly equivalent success to
  those of males, partially discounting recent findings that explicit behavior
  accompanies implicit gender bias, while bolstering the claim that implicit bias is
  cultural, rather than as a result of innate differences.
\end{abstract}

% \category{CR-number}{subcategory}{third-level}

% general terms are not compulsory anymore, 
% you may leave them out
% \terms
% term1, term2

\keywords
gender bias, software, open source, implicit bias

% Review Comments:
% - A more thorough discussion and analysis of the experimental set up would be appropriate.
% - Include more up-to-date numbers on enrollment in CS courses; there has been
% news about recent movement on that front, and certainly at top universities
% and in first CS courses
% -  include more related work; this is a greatly-studied question and there's
%    more that you can cite.
% - it is fairly well-known that females tend to be as competent as males when
%   you look at results, but they will just usually apply to things less often
%   when they don't feel completely qualified
% - The writing could also be tightened. As an example, the abstract contains a
%   number of different ideas which aren't linked together. When producing any
%   type of writing, try to make sure that each sentence is related to the ones
%   around it. There is reference to direct discriminatory behavior but I'd
%   expect that to be in the form of flaming emails. So I'd clarify what you
%   mean there. I think that what you mean is that peoples' pull requests get
%   rejected ostensibly due to being female.

\section{Problem and Motivation}

42\% of all doctoral degrees in science, technology, education, and
math (STEM) are earned by women, yet only 28\% of tenure-track faculty in these
fields are women \citep{womenengineering}. Why does this gender inequality exist?
Jackson, et al. claims that implicit bias and explicit behavior are main
factors of this finding \cite{implicitbias}. In this paper, I
contribute the first preliminary investigation of these factors of gender bias
by answering the following.

\begin{quote}
  \textbf{Research Question -} Does online collaborative interaction on GitHub result in
  explicit behavior which may contribute to the implicit biases of gender diversity?
\end{quote}

\section{Background and Related Work}

Many human cultures perceive female contribution to technical
fields as inappropriate due to religious or culturally constructed attitudes, with
negligible rationale \citep{elamin2010saudiwomen}. This has contributed to a gross imbalance in gender diversity among
the most technical computing and engineering fields, stymieing economic,
scientific, and innovative pursuits \citep{genderscience}

The resulting inefficiencies and creative
squander can be a severe handicap to any society, notwithstanding universal human rights
violations; By any modern deontological theory, human females are due equal
consideration as human males, ethically and professionally \citep{ethicsgender}.

Under-appreciation of an individual or group in a certain
discipline can discourage that group's contributions, further enforcing a
constructed bias for that discipline; software development is an example of such
a field \citep{genderscience}.

In 2005-2006, women accounted for only 20.6\% of awarded computer and information
science undergraduate degrees, according to the National Center for Education
Statistics (NCES), while women earned 52\% of all undergraduate
math and science degrees overall \citep{genderdiversitycomputing}.
Sax, et al. claims that there is some evidence of improvement, with female
undergraduate computer and information science students composing only 13.6\% of
their classes on average in 1971, the minimum recorded, and presently 18.2\%, a
25\% increase according to the NCES. The highest ratio of female students,
37.1\% was recorded in 1984 \citep{evolutionofwomen}.
This underrepresented population is a problem because of the great demand yet
decreasing supply of jobs in this field: By 2018, US universities will only
supply 52\% of the 1.4 million available jobs in computation and information
sciences \citep{genderdiversitycomputing}.

This gender imbalance could be caused by many troubling subtleties
in societies, one potential being the differences, if any, in interactions
between females and males, via any medium \citep{implicitbias}.

These interactions include asynchronous web-based software development
collaboration. If there are significant explicit behavioral differences in the
treatment given to males and females, revealing evidence in the overall success of
collaborative project contributions on GitHub, an open-source software
repository host, may be one indication. The ability to mine this information
presents an interesting ecosystem to observe real user behavior to validate current
literature on implicit bias.

\section{Approach and Uniqueness}

GitHub does not record gender information about its users. Automated gender
determination has recently been achieved with accuracy. \cite{VasilescuIWC13}
Vasilescu (2014), granted access to user data mined from GitHub's GHTorrent in November 2014,
along with the gender of all GitHub users up to that date.

Of 2.9 million usernames, 1.8 million were ambiguous and thus
not used for the purposes of this study. Of the remaining, 91 thousand were
female users and 958 thousand were male.

GitHub's contributions are organized in pull requests, complete with a single
user who created the pull request and a status. The status may be open, closed,
or merged. Merged pull requests are seen by this study as a success, and closed pull
requests are determined to be non-successful.

In tandem with GHTorrent's GitHub data store to
access user pull request history, the user-gender data allowed the computation of
the ratio of merge success for each gender. This level of intricate user interaction
between varying diverse people groups has never been closely investigated until
now. This single data metric encourages a more complete examination to better
understand the results.

The significance in the difference of female to male success is not the most
significant contribution: this study will pioneer this process for user data on
other public collaborative groups online.
On GitHub, there is still much data mining to do to achieve a more complete
picture of why the ecosystem is underrepresented. Further analysis of other
collaboration metrics is advised.

In this study, I use only one source of real-world behavioral data.
Many other collaborative, public environments exist on the Internet, complete
with programmatic access: Wikipedia, Stack Overflow, etc. These other
communities represent different mutable, mutually inclusive subgroups of the global
society and can be used to understand behavioral tendencies of those subgroups
towards gender perception.

Due to the politically volatile nature of this and other related data on gender
or diversity distributions in a service's use, many organizations which value
public opinion are apprehensive about advertising gender inequalities. These new data
mining methods advance the reach of social science into non-obtrusive public
ecosystems of the internet without exposing a company's prior knowledge of a
potential issue.

\section{Results and Contributions}

\begin{center}
  \begin{tabular}{l|l|l|l}
     & Pull Requests & Merged & Closed \\ \hline
     Female & \num{19e4} & \num{88e3} & \num{95e3} \\ \hline
     Male & \num{36e5} & \num{16e5} & \num{17e5}
  \end{tabular}
\end{center}

I analyzed 3.8 million pull requests with determined gender information from
2.9 million different users. There were 197936 pull requests created by female contributers and 3651803 made by
male. Of the 197936 created by female, $44.91\%$ were successfully merged, while $44.37\%$ male-created pull requests
were merged successfully.

While female pull request contributions account for only $5.14\%$ of the total pull
requests submitted in this data set, it is surprising to find that the merge
success rate is nearly equivalent to male success rate. Unequal gender
distribution found in technical fields may yet be worsened by unequal treatment;
however, the public proceedings of GitHub users, at least until November 2014,
shows no bias in female collaborative success, despite the 8.71\% female
user base represented by that data.

Speculations as to the cause of these results can lead to more detailed studies,
as mentioned previously;

Perhaps female software developers are bolstered by hardships faced as a severe
minority, thus increasing their average quality of contribution. If female
contributions are of greater quality than that of males, any discriminatory
action against female contributors may be statistically nuetralized. Static
analysis tools could be used on the commit content of each pull request gathered
in this study to determine if a quality-of-work bias occurs between genders;
\cite{staticanalysis} Determining a causal relationship to the skewed merge success rate may be a
challenge, however. Such a study would necessarily relate merge success rate to
pull request quality, with gender as a control.

Due to the widespread knowledge of gender inequality in software development,
female developers may choose to anonymize their gender online. This bias would
be very difficult to detect; in this study, users of GitHub whose usernames are
not indicative of a gender were discarded. These users of unknown gender may be
an interesting "third gender" to consider in all of the above data queries. By
analyzing other indicators, such as the contribution quality, mentioned above,
significant findings may be able to be determined for those users who choose to
avoid gender identification.

% \appendix
% \section{Appendix Title}
% This is the text of the appendix, if you need one.

% \acks

% Acknowledgments, if needed.

% We recommend abbrvnat bibliography style.

\bibliographystyle{abbrvnat}

\bibliography{ajkofink}

\end{document}
